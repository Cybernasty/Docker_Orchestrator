\subsection{Product Backlog}
The Product Backlog for the Docker Orchestrator platform encompasses functional and technical features across 5 months of development, organized into 3 major releases and 10 sprints. Each item is represented as a User Story following the format: "As a [role], I want [feature] so that [benefit]."

\subsubsection{Release Planning Overview}
\begin{itemize}
    \item \textbf{Release 1 (Sprints 1-5):} Core Platform Foundation and Features - Backend, Authentication, Container Management, Real-time Metrics, and User Interface
    \item \textbf{Release 2 (Sprints 6-8):} Deployment Part - Kubernetes, CI/CD Pipeline, ArgoCD GitOps, Monitoring (Grafana \& Prometheus), and High Availability
    \item \textbf{Release 3 (Sprints 9-10):} Security Part - Vulnerability Scanning (Trivy, SonarQube, OWASP ZAP), Secret Management (HashiCorp Vault), Security Policies, and Compliance
\end{itemize}

\begin{longtable}{|c|c|p{8cm}|c|c|}
\caption{User Stories and Task Breakdown by Sprint} \label{tab:userstories_sprint} \\
\hline
\textbf{Sprint} & \textbf{ID} & \textbf{User Story} & \textbf{Priority} & \textbf{Duration} \\
\hline
\endfirsthead
\multicolumn{5}{c}%
{{\bfseries \tablename\ \thetable{} -- continued from previous page}} \\
\hline
\textbf{Sprint} & \textbf{ID} & \textbf{User Story} & \textbf{Priority} & \textbf{Duration} \\
\hline
\endhead
\hline \multicolumn{5}{r}{{Continued on next page}} \\
\endfoot
\hline
\endlastfoot

% ============ RELEASE 1: CORE PLATFORM FOUNDATION AND FEATURES ============
\multicolumn{5}{|c|}{\cellcolor{blue!25}\textbf{RELEASE 1: Core Platform Foundation and Features}} \\
\hline

% Sprint 1 - Backend Foundation
1 & US-01 & As a backend developer, I want to build a RESTful API with Express.js so that the frontend can communicate with Docker services & High & 4 days \\ \hline
1 & US-02 & As a backend developer, I want to integrate MongoDB with replica sets so that data is persisted reliably & High & 3 days \\ \hline
1 & US-03 & As a developer, I want to implement JWT-based authentication so that users can securely access the platform & High & 3 days \\ \hline
1 & US-04 & As a backend developer, I want to create error handling middleware so that API errors are consistent and informative & Medium & 2 days \\ \hline

% Sprint 2 - User Management and Frontend Foundation
2 & US-05 & As an admin, I want role-based access control (Admin, Operator, Viewer) so that user permissions are properly managed & High & 4 days \\ \hline
2 & US-06 & As a frontend developer, I want to create a responsive React application with Tailwind CSS so that users have an intuitive interface & High & 5 days \\ \hline
2 & US-07 & As a user, I want a login page with form validation so that I can securely authenticate to the system & High & 3 days \\ \hline
2 & US-08 & As an admin, I want a user management interface to create, edit, and delete users so that I can control access & High & 2 days \\ \hline

% Sprint 3 - Core Container Management
3 & US-09 & As a user, I want to view all running and stopped Docker containers so that I can monitor my infrastructure & High & 3 days \\ \hline
3 & US-10 & As an operator, I want to start, stop, and restart containers so that I can manage container lifecycle & High & 4 days \\ \hline
3 & US-11 & As an operator, I want to view container logs in real-time so that I can troubleshoot issues & High & 4 days \\ \hline
3 & US-12 & As a user, I want a dashboard showing container statistics so that I have an overview of my system & High & 3 days \\ \hline

% Sprint 4 - Image and Volume Management
4 & US-13 & As a user, I want to view all Docker images so that I can see available images in my registry & Medium & 3 days \\ \hline
4 & US-14 & As an operator, I want to pull images from registries and build custom images so that I can deploy applications & High & 4 days \\ \hline
4 & US-15 & As a user, I want to manage Docker volumes so that I can persist container data & Medium & 3 days \\ \hline
4 & US-16 & As an admin, I want to create containers from images with custom configurations so that I can deploy services & High & 4 days \\ \hline

% Sprint 5 - Monitoring and User Experience
5 & US-17 & As a user, I want real-time CPU, memory, and network metrics for containers so that I can monitor resource usage & High & 5 days \\ \hline
5 & US-18 & As a user, I want visual charts and sparklines showing metric trends so that I can identify performance issues & High & 4 days \\ \hline
5 & US-19 & As a user, I want a profile page to update my information and change my password so that I can manage my account & Medium & 2 days \\ \hline
5 & US-20 & As a user, I want application settings (notifications, appearance, Docker registry) so that I can customize my experience & Medium & 2 days \\ \hline
5 & US-21 & As a user, I want terminal/shell access to containers so that I can execute commands for debugging & Medium & 2 days \\ \hline

% ============ RELEASE 2: DEPLOYMENT PART ============
\multicolumn{5}{|c|}{\cellcolor{green!25}\textbf{RELEASE 2: Deployment Part - Kubernetes and CI/CD}} \\
\hline

% Sprint 6 - Containerization and Kubernetes Setup
6 & US-22 & As a DevOps engineer, I want to containerize the application with Docker so that it can be deployed consistently & High & 3 days \\ \hline
6 & US-23 & As a DevOps engineer, I want to create Kubernetes manifests (Deployments, Services, ConfigMaps) for all services so that the platform can run on Kubernetes & High & 5 days \\ \hline
6 & US-24 & As a DevOps engineer, I want to configure Kubernetes Secrets for sensitive data so that credentials are securely managed & High & 2 days \\ \hline
6 & US-25 & As a DevOps engineer, I want Kubernetes Ingress configured with SSL/TLS so that the application is accessible via HTTPS & High & 3 days \\ \hline

% Sprint 7 - CI/CD Pipeline
7 & US-26 & As a DevOps engineer, I want a CI/CD pipeline with GitHub Actions for automated testing so that code quality is maintained & High & 4 days \\ \hline
7 & US-27 & As a DevOps engineer, I want automated Docker image building and pushing to registry so that deployments are streamlined & High & 3 days \\ \hline
7 & US-28 & As a DevOps engineer, I want ArgoCD installed and configured for GitOps so that Kubernetes deployments are declarative & High & 5 days \\ \hline
7 & US-29 & As a DevOps engineer, I want automated deployment workflows triggered by Git commits so that changes are deployed automatically & High & 2 days \\ \hline

% Sprint 8 - Monitoring and High Availability
8 & US-30 & As a DevOps engineer, I want Prometheus installed and configured so that metrics are collected from all services & High & 4 days \\ \hline
8 & US-31 & As a DevOps engineer, I want Grafana dashboards for monitoring containers, pods, and system resources so that I can visualize metrics & High & 4 days \\ \hline
8 & US-32 & As a DevOps engineer, I want alerting rules in Prometheus so that I am notified of critical issues & Medium & 2 days \\ \hline
8 & US-33 & As a DevOps engineer, I want Horizontal Pod Autoscaling (HPA) configured so that the platform scales based on resource usage & High & 3 days \\ \hline
8 & US-34 & As a DevOps engineer, I want MongoDB configured with replica sets and authentication so that the database is highly available & High & 3 days \\ \hline

% ============ RELEASE 3: SECURITY PART ============
\multicolumn{5}{|c|}{\cellcolor{red!25}\textbf{RELEASE 3: Security Part - Scanning, Vault, Policies, and Compliance}} \\
\hline

% Sprint 9 - Security Scanning and Secret Management
9 & US-35 & As a security engineer, I want SonarQube integrated for code quality and vulnerability analysis so that code security is ensured & High & 3 days \\ \hline
9 & US-36 & As a security engineer, I want Trivy integrated for container image vulnerability scanning so that Docker images are secure & High & 3 days \\ \hline
9 & US-37 & As a security engineer, I want OWASP ZAP integrated for dynamic application security testing so that runtime vulnerabilities are detected & High & 4 days \\ \hline
9 & US-38 & As a security engineer, I want HashiCorp Vault deployed for secure secret management so that sensitive data is encrypted and protected & High & 4 days \\ \hline
9 & US-39 & As a DevOps engineer, I want application secrets migrated to Vault so that credentials are centrally managed and rotated & High & 2 days \\ \hline

% Sprint 10 - Security Policies, Network Security, and Compliance
10 & US-40 & As a security engineer, I want automated security scans in the CI/CD pipeline so that vulnerabilities are caught before deployment & High & 2 days \\ \hline
10 & US-41 & As a security engineer, I want container security policies (privileged mode, read-only filesystem, no-new-privileges) implemented so that containers follow security best practices & High & 3 days \\ \hline
10 & US-42 & As a security engineer, I want Kubernetes Network Policies configured so that pod-to-pod communication is restricted and segmented & High & 3 days \\ \hline
10 & US-43 & As a security engineer, I want a security dashboard showing scan results, CVE details, and security metrics so that vulnerabilities can be tracked and prioritized & High & 4 days \\ \hline
10 & US-44 & As a security engineer, I want security audit logs for all user actions and API calls so that compliance requirements are met & Medium & 2 days \\ \hline
10 & US-45 & As a user, I want to view security scan results for my containers so that I can address identified vulnerabilities & Medium & 2 days \\ \hline

\end{longtable}

\subsubsection{Definition of Done (DoD)}
For each User Story to be considered complete, the following criteria must be met:
\begin{itemize}
    \item Code is written, reviewed, and merged to the main branch
    \item Unit tests are written with at least 70\% code coverage
    \item Integration tests pass successfully
    \item Security scans show no critical vulnerabilities
    \item Documentation is updated (README, API docs, deployment guides)
    \item Feature is deployed to staging environment and tested
    \item Product Owner approves the implementation
\end{itemize}

\subsubsection{Sprint Velocity and Capacity}
\begin{itemize}
    \item Sprint Duration: 2 weeks (10 working days)
    \item Team Capacity: Approximately 12-15 story points per sprint
    \item Story Point Estimation: 1 day = 1 story point
    \item Buffer: 20\% reserved for bug fixes, technical debt, and unforeseen issues
\end{itemize}

\subsubsection{Release Milestones and Deliverables}
\begin{itemize}
    \item \textbf{Release 1 (End of Sprint 5 - Month 2.5):}
    \begin{itemize}
        \item Functional web application with authentication and RBAC
        \item Complete container management (CRUD operations)
        \item Real-time monitoring dashboard with metrics
        \item Image and volume management
        \item User management and profile settings
        \item Demo-ready platform for stakeholders
    \end{itemize}
    
    \item \textbf{Release 2 (End of Sprint 8 - Month 4):}
    \begin{itemize}
        \item Platform containerized and running on Kubernetes
        \item CI/CD pipeline with automated testing and deployment
        \item ArgoCD GitOps setup for declarative deployments
        \item Monitoring infrastructure with Prometheus and Grafana
        \item High availability with autoscaling and health monitoring
        \item Production-ready infrastructure with observability
    \end{itemize}
    
    \item \textbf{Release 3 (End of Sprint 10 - Month 5):}
    \begin{itemize}
        \item Integrated security scanning (SonarQube, Trivy, OWASP ZAP)
        \item HashiCorp Vault for secret management and encryption
        \item Container security policies and network segmentation
        \item Security dashboard with vulnerability tracking and metrics
        \item Audit logs and compliance features
        \item Fully secured, compliant, production-grade platform
    \end{itemize}
\end{itemize}

\subsubsection{Priority Levels Explained}
\begin{itemize}
    \item \textbf{High:} Critical features required for the release, blocking dependencies
    \item \textbf{Medium:} Important features that enhance functionality but are not blockers
    \item \textbf{Low:} Nice-to-have features that can be deferred if needed
\end{itemize}

\subsubsection{Technical Debt and Maintenance}
Each sprint allocates time for:
\begin{itemize}
    \item Bug fixes from previous sprints
    \item Code refactoring and optimization
    \item Dependency updates and security patches
    \item Performance improvements
    \item Technical documentation updates
\end{itemize}

